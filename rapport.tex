% !TEX encoding = IsoLatin

\documentclass[12pt,ULlof,ULlot]{ULrapport}

% Chargement des packages supplementaires (si absent de la classe)
\usepackage[ansinew]{inputenc}
\usepackage[autolanguage]{numprint}
\usepackage{icomma}
\usepackage{hyperref}
\usepackage{placeins}
\usepackage{array}
\usepackage{amsmath}
\usepackage{longtable}
%\usepackage{amsmath}
%\usepackage[options]{nom_du_package}

% Definition d'une commande pour presenter des cellules multilignes dans un tableau
\newcommand{\cellulemultiligne}[1]{\begin{tabular}{@{}c@{}}#1\end{tabular}}

% Definition de colonnes en mode paragraphe avec alignement ajustable
% Cette definition requiert le chargement du package "array"
%    - alignement horizontal, parametre #1 : - \raggedright (aligne a gauche)
%                                            - \centering (centre)
%                                            - \raggedleft (aligne a droite)
%    - alignement vertical, parametre #2 : - p (aligne en haut)
%                                          - m (centre)
%                                          - b (aligne en bas)
%    - largeur, parametre #3 : longueur
\newcolumntype{Z}[3]{>{#1\hspace{0pt}\arraybackslash}#2{#3}}
\newcolumntype{P}[1]{>{\raggedright\arraybackslash}p{#1}}

% Definitions des parametres de la page titre
\TitreProjet{Stage �t� 2014}                         % Titre du projet
\TitreRapport{Rapport}                       % Titre du rapport
\Destinataire{D�partement de G�nie Logiciel}         % Nom(s) du destinataire
\NumeroEquipe{01}                                     % Numero de l'equipe
\NomEquipe{Aristote Diasonama}                               % Nom de l'equipe
\TableauMembres{%                                     % Tableau des membres de l'equipe
   \hline        % matricule & nom & \\\hline
   000\,000\,000  & Aristote Diasonama         & G�nie Logiciel\\\hline        % matricule & nom & \\\hline
}
\DateRemise{03 09 2014}                           % Date de remis


% Contenu de l'historique des versions
\HistoriqueVersions{%                        % version & date & description \\\hline
       &  24 aout 2013 & Cr�ation du document \\ \hline
	VF   & 03 septembre 2014 & Rapport version finale \\ \hline
%   1.3   & 14 janvier 2011 & reformatage de l'exemple, changements � l'organisation des figures\\\hline
%   1.3.1   & 24 novembre 2011 & matricules � 9 chiffres, titre du rapport\\\hline
}


% Corps du document

\begin{document}

%   Chapitres
% !TEX encoding = IsoLatin

%
% Chapitre "Introduction"
%

\chapter{Introduction}
\label{s:intro}

En �t� 2014, je faisais d�j� mon premier stage en Informatique � Wanted Technologies travaillant sur les outils internes pour la g�n�ration de rapport ainsi que quelques corrections des bugs dans l'API. � l'issu de ce stage riche en exp�rience inoubliable, j'�tais convaincu que mon choix de cursus acad�mique correspondait bien � ce que je voulais faire dans la vie.
Cet �t� je suis retourn� dans la meme entreprise, � l'occurrence Wanted Technologies mais cette fois, dans une autre �quipe, l'�quipe User Interface (UI) . C'est donc de cette experience v�cue dans l'�quipe UI qu'il sera question dans les prochaines lignes de ce rapport. 


\section{Wanted Technologies}
\label{about_wanted}

WANTED Technologies est une entreprise informatique cr�� en 1999 qui fournit de l'information de veille commerciale en temps r�el pour le march� du recrutement. La soci�t� a son si�ge social � Qu�bec, au Canada, et maintient une filiale am�ricaine ayant ses bureaux principaux � New York. Elle accumule, depuis octobre 2002, les d�tails associ�s aux offres d'emploi et maintient actuellement une base de donn�es surpassant un milliard d'offres d'emploi uniques.
Ses clients qui proviennent des secteurs tels que  les ressources humaines, les services de recrutement, les m�dias et les gouvernements utilisent WANTED Analytics, son logiciel principal,  pour identifier et prioriser les pistes de vente, cerner les tendances �conomiques, analyser les activit�s de la concurrence, estimer les conditions �conomiques futures ainsi qu'identifier des candidats pour des postes difficiles � combler.  \cite{REF01}



%!TEX encoding = IsoLatin

%
% Chapitre "Bibliographie"
%

\begin{thebibliographyUL}{99} % remplacer le "{9}" par "{99}" lorsque le nombre de references
                              % requiert 2 caracteres (>= 10 references)

\bibitem{REF01} Wanted Technologies. \emph{A propos de nous}, [En ligne]. \url{https://www.wantedanalytics.com/fr/a-propos/a-propos-de-nous} (Page consult�e le 24 aout 2014)
 \end{thebibliographyUL}






% !TEX encoding = IsoLatin

%
% Chapitre "Environnement de Travail"
%

\chapter{Environnement de travail}
\label{s:env_travail}
\section{Organisation de l'entreprise}
\label{sec:organisation_equipes}
Wanted poss�de des �quipes diversifi�es d'employ�s dans ses bureaux de Qu�bec. Aux cot�s des membres de la direction de l'entreprise, nous retrouvons les �quipes de marketing, d'assurance qualit�, de livraison produit, les �quipes de d�veloppeurs. Il existe une communication permanente entre ces diff�rentes �quipes ce qui permet certainement le bon fonctionnement de l'entreprise. 
Durant mon stage, je faisais �videmment partie de l'�quipe technique, l'�quipe des d�veloppeurs.
En effet, les d�veloppeurs � Qu�bec sont organis�s en 5 principales �quipes comme suit:

1. L'acquisition: Cette �quipe con�oit, entretient et maintient les logiciels qui permettent � Wanted Technologies d'obtenir les donn�es sur le march� d'emploi. Les logiciels con�us par cette �quipe, vont rechercher des donn�es sur le march� d'emploi et les mettre � la disposition des bases de donn�es de Wanted.

2. L'�quipe de base de donn�es: Cette �quipe est charg�e de la reception, l'analyse, l'organisation et la sauvegarde des donn�es que les logiciels de l'�quipe d'acquisition ram�nent � Wanted.
Les donn�es renvoy�es par l'�quipe d'acquisition sont � l'�tat pur, l'�quipe de base de donn�es va donc les traiter et les organiser de mani�re � ce qu'elles deviennent compr�hensibles et utilisables par Wanted et ses clients.

3. Le middleware: Cette �quipe est celle qui s'occupe de l'API Wanted. L'API permet aux clients de Wanted de consulter la vaste quantit� des donn�es sur le march� d'emploi qui ont �t� bien trait�es et sauvegard�es par l'�quipe de base de donn�es. L'API offre des services qui facilitent donc la consultation des donn�es par les clients.

4. L'�quipe d'interface utilisateur (UI): Cette �quipe est quant � elle charg�e de pr�senter les donn�es de Wanted via une interface web. L'�quipe UI pr�sente les donn�es de mani�re � ce qu'elles soient faciles � lire et comprendre pour les clients qui ne sont pas n�cessairement int�ress�s par les donn�es � l'�tat brut. Cette �quipe mod�lise les 
graphiques qui permettent ainsi de bien assimiler ce qui se passe dans le march� d'emploi.

5. L'�quipe d'assurance qualit� (QA): La derni�re et non la moindre, cette �quipe est charg�e d'assurer que les produits livr�s aux clients respectent bien les standards de qualit� �tablis par Wanted. Les membres de cette �quipe effectue des tests avanc�s sur les produits avant la livraison comme apr�s la livraison. Cela permet de s'assurer que les produits livr�s par Wanted sont toujours de tr�s haute qualit�.




\section{Mon Role au sein de l'�quipe}
\label{sec:about_wanted}

Sous la supervision de Gaetan Corneau, Directeur R\&D � Wanted, j'ai pass� mon stage au sein de l'�quipe middleware. 
Avec l'aide de mon chef d'�quipe, Jean-Sebastien Vachon, et tous les membres de l'�quipe, mon int�gration au sein de l'�quipe a plut�t �t� tr�s facile. J'ai eu � participer � diff�rents types de projets au sein du middleware\ref{s:taches}.
Bien que stagiaire, des t�ches de d�veloppement touchant directement � une nouvelle version majeure de Wanted Analytics m'ont �t� confi�es.





%   Annexes
%\appendix
%\input{tex/liste_sig_acr}

\end{document}
% Fin du document

