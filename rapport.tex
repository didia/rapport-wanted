% !TEX encoding = IsoLatin

\documentclass[12pt,ULlof,ULlot]{ULrapport}

% Chargement des packages supplementaires (si absent de la classe)
\usepackage[ansinew]{inputenc}
\usepackage[autolanguage]{numprint}
\usepackage{icomma}
\usepackage{hyperref}
\usepackage{placeins}
\usepackage{array}
\usepackage{amsmath}
\usepackage{longtable}
\usepackage{textcomp}
%\usepackage{amsmath}
%\usepackage[options]{nom_du_package}

% Definition d'une commande pour presenter des cellules multilignes dans un tableau
\newcommand{\cellulemultiligne}[1]{\begin{tabular}{@{}c@{}}#1\end{tabular}}

% Definition de colonnes en mode paragraphe avec alignement ajustable
% Cette definition requiert le chargement du package "array"
%    - alignement horizontal, parametre #1 : - \raggedright (aligne a gauche)
%                                            - \centering (centre)
%                                            - \raggedleft (aligne a droite)
%    - alignement vertical, parametre #2 : - p (aligne en haut)
%                                          - m (centre)
%                                          - b (aligne en bas)
%    - largeur, parametre #3 : longueur
\newcolumntype{Z}[3]{>{#1\hspace{0pt}\arraybackslash}#2{#3}}
\newcolumntype{P}[1]{>{\raggedright\arraybackslash}p{#1}}

% Definitions des parametres de la page titre
\TitreProjet{Stage �t� 2015}                         % Titre du projet
\TitreRapport{Rapport}                       % Titre du rapport
\Destinataire{D�partement de G�nie Logiciel}         % Nom(s) du destinataire
\NumeroEquipe{01}                                     % Numero de l'equipe
\NomEquipe{Aristote Diasonama}                               % Nom de l'equipe
\TableauMembres{%                                     % Tableau des membres de l'equipe
   \hline        % matricule & nom & \\\hline
   000\,000\,000  & Aristote Diasonama         & G�nie Logiciel\\\hline        % matricule & nom & \\\hline
}
\DateRemise{09 09 2015}                           % Date de remis


% Contenu de l'historique des versions
\HistoriqueVersions{%                        % version & date & description \\\hline
       &  24 aout 2015 & Cr�ation du document \\ \hline
	VF   & 09 septembre 2015 & Rapport version finale \\ \hline
%   1.3   & 14 janvier 2011 & reformatage de l'exemple, changements � l'organisation des figures\\\hline
%   1.3.1   & 24 novembre 2011 & matricules � 9 chiffres, titre du rapport\\\hline
}


% Corps du document

\begin{document}

%   Chapitres
% !TEX encoding = IsoLatin

%
% Chapitre "Introduction"
%

\chapter{Introduction}
\label{s:intro}

En �t� 2014, je faisais d�j� mon premier stage en Informatique � Wanted Technologies travaillant sur les outils internes pour la g�n�ration de rapport ainsi que quelques corrections des bugs dans l'API. � l'issu de ce stage riche en exp�rience inoubliable, j'�tais convaincu que mon choix de cursus acad�mique correspondait bien � ce que je voulais faire dans la vie.
Cet �t� je suis retourn� dans la meme entreprise, � l'occurrence Wanted Technologies mais cette fois, dans une autre �quipe, l'�quipe User Interface (UI) . C'est donc de cette experience v�cue dans l'�quipe UI qu'il sera question dans les prochaines lignes de ce rapport. 


\section{Wanted Technologies}
\label{about_wanted}

WANTED Technologies est une entreprise informatique cr�� en 1999 qui fournit de l'information de veille commerciale en temps r�el pour le march� du recrutement. La soci�t� a son si�ge social � Qu�bec, au Canada, et maintient une filiale am�ricaine ayant ses bureaux principaux � New York. Elle accumule, depuis octobre 2002, les d�tails associ�s aux offres d'emploi et maintient actuellement une base de donn�es surpassant un milliard d'offres d'emploi uniques.
Ses clients qui proviennent des secteurs tels que  les ressources humaines, les services de recrutement, les m�dias et les gouvernements utilisent WANTED Analytics, son logiciel principal,  pour identifier et prioriser les pistes de vente, cerner les tendances �conomiques, analyser les activit�s de la concurrence, estimer les conditions �conomiques futures ainsi qu'identifier des candidats pour des postes difficiles � combler.  \cite{REF01}


% !TEX encoding = IsoLatin

%
% Chapitre "Environnement de Travail"
%

\chapter{Environnement de travail}
\label{s:env_travail}
\section{Organisation de l'entreprise}
\label{sec:organisation_equipes}
Wanted poss�de des �quipes diversifi�es d'employ�s dans ses bureaux de Qu�bec. Aux cot�s des membres de la direction de l'entreprise, nous retrouvons les �quipes de marketing, d'assurance qualit�, de livraison produit, les �quipes de d�veloppeurs. Il existe une communication permanente entre ces diff�rentes �quipes ce qui permet certainement le bon fonctionnement de l'entreprise. 
Durant mon stage, je faisais �videmment partie de l'�quipe technique, l'�quipe des d�veloppeurs.
En effet, les d�veloppeurs � Qu�bec sont organis�s en 5 principales �quipes comme suit:

1. L'acquisition: Cette �quipe con�oit, entretient et maintient les logiciels qui permettent � Wanted Technologies d'obtenir les donn�es sur le march� d'emploi. Les logiciels con�us par cette �quipe, vont rechercher des donn�es sur le march� d'emploi et les mettre � la disposition des bases de donn�es de Wanted.

2. L'�quipe de base de donn�es: Cette �quipe est charg�e de la reception, l'analyse, l'organisation et la sauvegarde des donn�es que les logiciels de l'�quipe d'acquisition ram�nent � Wanted.
Les donn�es renvoy�es par l'�quipe d'acquisition sont � l'�tat pur, l'�quipe de base de donn�es va donc les traiter et les organiser de mani�re � ce qu'elles deviennent compr�hensibles et utilisables par Wanted et ses clients.

3. L'�quipe datascience: Cette �quipe est charg�e d'interpreter les donn�es recueillies et stock�es pour en trouver des variables ou des corr�lations utiles aux analyses effectu�es par Wanted.  

4. Le middleware: Cette �quipe est celle qui s'occupe de l'API Wanted. L'API permet aux clients de Wanted de consulter la vaste quantit� des donn�es sur le march� d'emploi qui ont �t� bien trait�es et sauvegard�es par l'�quipe de base de donn�es. L'API offre des services qui facilitent donc la consultation des donn�es par les clients.

5. L'�quipe d'interface utilisateur (UI): Cette �quipe est quant � elle charg�e de pr�senter les donn�es de Wanted via une interface web. L'�quipe UI pr�sente les donn�es de mani�re � ce qu'elles soient faciles � lire et comprendre pour les clients qui ne sont pas n�cessairement int�ress�s par les donn�es � l'�tat brut. Cette �quipe mod�lise les 
graphiques qui permettent ainsi de bien assimiler ce qui se passe dans le march� d'emploi.

6. L'�quipe d'assurance qualit� (QA): La derni�re et non la moindre, cette �quipe est charg�e d'assurer que les produits livr�s aux clients respectent bien les standards de qualit� �tablis par Wanted. Les membres de cette �quipe effectue des tests avanc�s sur les produits avant la livraison comme apr�s la livraison. Cela permet de s'assurer que les produits livr�s par Wanted sont toujours de tr�s haute qualit�.




\section{Mon Role au sein de l'�quipe}
\label{sec:about_wanted}

Sous la supervision de Gaetan Corneau, Directeur R\&D � Wanted, j'ai pass� mon stage au sein de l'�quipe UI. 
Ayant d�j� effectu� un stage � Wanted auparavant, l'int�gration a �t� tr�s facile.
Au sein de cette �quipe, j'ai eu � participer � plusieurs projets tr�s certains dont plusieurs sont actuellement en production. 
Dans les prochaines lignes, nous explorerons en d�tail ces diff�rentes taches.



% !TEX encoding = IsoLatin

%
% Chapitre "Environnement de Travail"
%

\chapter{T�ches effectu�es}
\label{s:taches}
Durant mon stage au sein de Wanted, j'ai eu � effectuer plusieurs t�ches en relation avec ma position dans l'�quipe d'interface utilisateur. Dans ce chapitre, je vais d�crire quatre de ces t�ches qui ont toutes �t� pleines de d�fis et d'opportunit�s d'apprentissage:
\begin{enumerate}
 \item World Map Popup. \ref{sec:world_map}.
 \item Wanted Analytics Connector: Icims Integration. \ref{sec:icims}
 \item Alternate Location Highlights \ref{sec:alternate_location}
 \item Bullhorn Integration \ref{sec:bullhorn}.

\end{enumerate}

\section{World Map Popup}
\label{sec:world_map}
\subsection {Mise en contexte}
La world map est une de nombreuses features que notre produit Wanted Analytics\texttrademark. Elle permet donc � nos clients d'avoir une vue d'ensemble du march� d'emploi dans les diff�rents pays que nous supportons. Dans notre application Web, elle se pr�sente comme une page avec une map du monde o� tous les pays que nous supportons sont �tiquet�s du nombre d'offres d'emploi actuellement disponibles dans ces pays. \ref{fig:worldmap-before}
\subsection{Demande du client}
Le client voulait �tre en mesure d'avoir acc�s non seulement au nombre d'offres d'emploi ouvertes mais aussi � une estimation du nombre des candidats potentiels par pays pour une recherche donn�e � travers la world map. Il voulait �tre en mesure de voir � travers la meme world map, pour chaque pays, les top 5 des aires urbaines o� il y a plus d'offres d'emploi ou de candidats potentiels.
\subsection{La solution}
La r�ponse � cette demande a �t� apport�e sous forme d'une release nomm�e World Map Popup. Deux �quipes ont pris part au d�veloppement de cette feature. L'�quipe Middleware s'est charg� de la partie backend qui consistait � exposer les donn�es via des services REST.  Du cot� UI, la solution �tait d'ajouter une nouvelle vue pour afficher le nombre des candidats potentiels par pays. Le client sera donc pr�sent� un s�lecteur de vue qui lui permettra de naviguer d'une vue � l'autre au besoin. 
Concernant la demande sur le top 5 des aires urbaines, nous avons ajout� un popup qui affiche le top 5 des aires urbaines comme demand� par le client. En passant sa souris sur un pays voulu, le client se vera affich� ce popup avec toutes les info dont il a besoin. \ref{fig:worldmap-after}
 
\section{Icims Integration}
\label{sec:icims}
\subsection{Mise en contexte}
Un autre produit de Wanted est le Wanted Analytics Connector (WAC). Le WAC est une extension de navigateur web. Cette extension qui est disponible sur Google Chrome, Mozilla Firefox et Internet Explorer, permet au client qui l'utilise d'obtenir des informations sur des offres d'emploi qu'il visite quand ces derniers sont publi�es sur des sites reconnus par Wanted. Pour ce faire le WAC analyse la structure de la page visit�e pour retrouver des mots cl�s sur l'offre d'emploi qui seront ensuite envoy�s au serveur de Wanted Analytics pour g�n�rer l'analyse de cette offre.
\subsection{Demande du client}
Le client voulait donc ajouter le support d'icims au WAC. Icims est une plateforme web permettant aux recruteurs de g�rer et publier des offres d'emplois. La solution devait donc permettre aux clients d'utiliser le WAC sur Icims.
\subsection{La solution}
La solution pour ce probl�me a �t� direct. Nous avons analys� la structure de la plateforme web d'icims et nous avons �t� en mesure d'extraire les informations n�cessaires pour ajouter le support du WAC.

\section{Alternate Location Highlights}
\label{sec:alternate_location}
Pour cette tache, il s'agissait d'apporter des modifications graphiques dans notre produit Web Wanted Analytics. Ces modifications devraient permettre de mettre en relief les locations d'int�r�ts particuliers  pour un utilisateur parmi plusieurs locations potentielles.
\section{Bullhorn Integration}
\label{sec:bullhorn}

Enfin, j'ai travaill� sur l'int�gration de notre application sur un autre site de strat�gie de recrutement. Comme dans le cas de Icims \ref{sec:icims}, il fallait aussi pour chaque offre d'emploi, afficher l'analyse fournie par le WAC. Sauf que pour ce cas, le WAC ne devrait pas �tre en forme d'un plugin mais plut�t int�gr� comme un iframe sur la page de l'offre d'emploi. Nous avons aussi apport� cette solution telle que pr�sent�e sur la figure \ref{fig:bullhorn}




% !TEX encoding = IsoLatin

%
% Chapitre "Environnement de Travail"
%

\chapter{Reflexion sur la formation pratique}
\label{s:formation_pratique}
Dans ce chapitre, ma formation pratique au sein de Wanted sera abord�e. Il s'agira notamment de la m�thode de d�veloppement utilis� pendant mon stage, l'organisation du travail au sein de mon �quipe du middleware et pour finir un bilan g�n�ral sur l'atteinte des objectifs fix�s pour le stage.

\section{M�thode de d�veloppement}
\label{sec:methode_developpement}
Pendant mon stage, j?ai �t� expos� � plusieurs m�thodes de travail. De mani�re generale, la methodologie de travail au sein de mon equipe etait la methodologie agile. Les taches  � effectuer �taient reparties aux �quipes selon leur domaine respectif.Ensuite au sein de l?equipe , chaque membre pouvait choisir la tache sur laquelle il voulait travailler. Le temps effectu� sur une tache est track� grace un syst�me de management des projects agiles.
Toutefois j?aimerai parler en particulier du processus des resolutions de bugs.

Pour un gros logiciel aussi utilis� que Wanted Analytics, les bugs font parties integrantes du travail journalier des developpeurs.

Les bugs sont principalement identifi�s par les clients, les analystes en qualit�, ainsi que les scripts de test de l?Application.
Une fois les bugs identif�s, ils sont transmis � l?�quipe concern�e. Ensuite un membre de l?equipe concern�e se mettra � corriger les bugs.
La correction se fait dans un premier temps dans un environnement de developpement. Une fois le bug corrig� et test� en developpement, la modification est effectu� dans un second environnement appel� staging. En staging, les analystes de qualit� vont alors tester si les bugs est vraiment resolu. Si la resolution du bug est accetp�, le travail est pass� en integration. Des tests vont s?effectuer en integration. Et la solution restera en integration jusqu?� ce qu?� la prochaine mise en production avec certainement d?autres corrections de bugs ou des nouvelles features d�j� en integration.
Au moment de la MEP, la correction va etre lanc� en production et plusieurs tests vont encore etre effectu�s pour s?assurer que la correction a bien �t� faite.
 
\section{Organisation du travail}
\label{sec:organisation_travail}
En tant que Stagiaire, les taches me furent attribu�s par le superviseur de mon �quipe. Nous commencerons par une r�union pour discuter du travail � accomplir. De la pr�sentation du probl�me � une petite analyse des possibles solutions.
Ensuite j?estime l?effort que �a me prendra d?accomplir la tache. L?effort est bas� sur l?experience personnnelle. L?effort s?exprime en nombre d?heures. La tache fait souvent partie d?un grand projet qui elle poss�de une �ch�ance. D?o� l?estimation de l?effort doit etre rationnel et reste dans le temps allou� pour le projet parent.
Dans la majorit� de cas, mes taches ont souvent prises une journ�e d?analyse profonde du probl�me qui est tr�s souvent d?une recherche des diff�rentes solutions possibles. 
Une fois ces deux �tapes pass�es, une solution est choisie et est implement�e. Des tests vont suivre pour valider la solution. Si les tests sont pass�es, le travail est termin�. Sinon on recommence avec l?analyse du probl�me et l?analyse des possibles solutions.

\section{Bilan du stage}
\label{sec:bilan_stage}

En effet l?objectif principal du stage tel que pr�sent� dans l?offre du stage �tait de concevoir et mettre en place un syst�me qui analyse les logs de services web afin de compiler des statistiques d�taill�es d'utilisation pour WANTED et ses clients.
Ce projet a pris la majeure partie de mon stage. Nous avons �t� deux stagiaires � travailler sur ce projet. 
Nous avons effectu� beacuoup d?analyse en amont, explor� beaucoup des pistes de solution avant de d�cider de la solution finale au projet. Ce projet m?a pris la moiti� du temps de stage.
Au d�part, j?�tais moins satisfait du projet � cause du temps qui �tait demand� pour l?analyse du probl�me. Cependant je me suis rendu compte de l?importance de cette �tape car cela nous a �vit� beaucoup des peines dans le future. Aujourd?hui le syst�me de rapport est en production et pour le moins que l?on puisse dire, je consid�re cette partie de stage r�ussie.
L?objectif secondaire tel que pr�sent� dans l?offre, c?�tait d?effectuer toutes taches connexes � mon poste. C?est vrai tel qu?�nonc�, cet objectif  a l?air plutot vague. Cependant, c?est dans ces taches connexes que j?ai plus trouv� du plaisir pendant mon stage.
Les t�ches que j?ai eues � faire pendant la deuxi�me partie de mon stage consistait principalement: R�soudre un bug dans Wanted Analytics, travailler sur un projet interne et ensuite travaille sur deux nouvelles features de la prochaine version de Wanted Analytics. Toutes ces taches ayant bien �t� effectu�es et compl�t�es, je consid�re cette partie de stage compl�tement r�ussie.

En effet l?objectif principal du stage tel que pr�sent� dans l?offre du stage �tait de concevoir et mettre en place un syst�me qui analyse les logs de services web afin de compiler des statistiques d�taill�es d'utilisation pour WANTED et ses clients.
Ce projet a pris la majeure partie de mon stage. Nous avons �t� deux stagiaires � travailler sur ce projet. 
Nous avons effectu� beacuoup d?analyse en amont, explor� beaucoup des pistes de solution avant de d�cider de la solution finale au projet. Ce projet m?a pris la moiti� du temps de stage.
Au d�part, j?�tais moins satisfait du projet � cause du temps qui �tait demand� pour l?analyse du probl�me. Cependant je me suis rendu compte de l?importance de cette �tape car cela nous a �vit� beaucoup des peines dans le future. Aujourd?hui le syst�me de rapport est en production et pour le moins que l?on puisse dire, je consid�re cette partie de stage r�ussie.
L?objectif secondaire tel que pr�sent� dans l?offre, c?�tait d?effectuer toutes taches connexes � mon poste. C?est vrai tel qu?�nonc�, cet objectif  a l?air plutot vague. Cependant, c?est dans ces taches connexes que j?ai plus trouv� du plaisir pendant mon stage.
Les t�ches que j?ai eues � faire pendant la deuxi�me partie de mon stage consistait principalement: R�soudre un bug dans Wanted Analytics, travailler sur un projet interne et ensuite travaille sur deux nouvelles features de la prochaine version de Wanted Analytics. Toutes ces taches ayant bien �t� effectu�es et compl�t�es, je consid�re cette partie de stage compl�tement r�ussie.



\section{Commentaires sur la recherche de stage, formation pr�-stage et Orientation de carri�re}
\label{sec:commentaires}

De la formation pr�-stage, je crois que deux cours ont �t� vraiment tr�s important pour le contenu de mon stage, le cours de g�nie logiciel orient� object (GLO-2004), ainsi que le cours du processus en genie logiciel. Le cours de GLO-2004 a mis les bases pour le langage Java que j?ai utilis� pendant mon stage. Aussi l?analyse que nous avons effectu�e pendant le projet de session pour ce cours m?a servi de base pour l?analyse que j?ai eue � effectuer au d�but de stage. D?autre part, le cours de processus en genie logiciel a �t� aussi d?une importance capitale surtout que c?est le cours qui m?a introduit au processus de d�veloppement en entreprise et sp�cialement au processus en g�nie logiciel. Ce qui a permis que je me sente d�j� � l?aise avec la terminologie et les concepts de base du processus que j?ai utilis� durant mon stage.
Avec ces deux connaissances acquises et une experience en tant que d�veloppeur dans un projet � l?universit� Laval, j?�tais sur que j?avais ce qu?il fallait pour me trouver un stage. La recherche de stage a �t� plutot facile pour moi. D�s que j?ai vu l?offre de Wanted, c?etait ce que j?ai cherch�. Tout s?est pass� correctement et j?ai eu mon stage rapidement.
Avec ce stage r�ussi, je n?ai jamais �t� aussi convaincu que c?est bien dans le g�nie logiciel que je veux faire carri�re. Cependant j?ai encore du mal � choisir dans quelle concentration je vais me lancer, je voudrais faire du logiciel industriel mais je pense que j?appr�cie enormement le d�veloppement web et mobile. Pour l?instant, pendant que je me donne encore le temps � reflechir sur ma concentration future, j?ai choisi de continuer � travailler dans tous les domaines o� l?on r�soud les probl�ms rencontr�s chaque jour.

%!TEX encoding = IsoLatin

%
% Chapitre "Reflexion th�orique"
%

\chapter{Reflexion sur la formation th�orique}
\label{s:reflexion_theorique}
La formation th�orique re�ue � l?universit� a �t� en majeure partie tr�s utile dans mon stage. S?il y a une recommandation que je devrais faire au directeur de programme, c?est d?essayer de trouver des voies et moyens pour encourager des activit�s pratiques au sein du programme d?informatique et du g�nie logiciel pendant lesquelles les �tudiants pourront appliquer toutes les th�ories r��ues en classe. La th�orie est bonne mais n?est pas suffisante pour am�liorer la capacit� de r�action face aux nouveaux probl�mes qui se posent. Il faudra encourgaer les activit�s pratiques de programmation tel que les hackatons.
En parlant de formation th�orique, j?aimerai un eclaircissement sur le concept d?architecture logicielle emergente de la part de mon professeur d?introduction au processus en genie logiciel.
Avant mon stage je voulais suivre un cours optionnel en d�veloppement web. Toutefois j?ai appris beaucoup sur le web pendant mon stage que je crois que je vais suivre un cours optionnel en d�veloppement mobile.




%!TEX encoding = IsoLatin

%
% Chapitre "Conclusion "
%

\chapter{Conclusion}
\label{s:conclusion}
Apr�s mon interview, j'�tais convaincu que mon stage serait une exp�rience enrichissante. Mon superviseur de stage, Mr Gaetan Corneau avait r�ussi en m'en persuader. A la fin du stage, je ne peux qu'affirmer que mon intuition �tait bien correcte!
En effet, avant mon stage, j'avais suivi le cours d'introduction aux processus de g�nie logiciel. De ce cours, j'avais appris diff�rentes �tapes principales du d�veloppement logiciel comme la phase de requis, d'analyse, d'implementation, d'integration, de tests, etc..
Durant mon stage, j'ai eu l'opportunit� de mettre pratique les acquis de ce cours et de me familiariser avec le processus de d�veloppement agile.
Mon aisance avec le langage de programmation Java s'est clairement am�lior�e gr�ce � l'aide pr�cieuse que toute mon �quipe m'a apport�e.
Maintenant, je vais continuer mon programme de baccalaur�at et continuer � apprendre et acqu�rir les connaissances n�cessaires qui vont me permettre de m'orienter vers le d�veloppement web et l'internet des objects qui sont les champs qui me fascinent pour l'instant.
 

% !TEX encoding = IsoLatin
\chapter{Annexe}

\begin{figure}[!ht]
\centering
\includegraphics[scale=1, width = 16cm, height = 12cm ]{fig/jasper.png}
\caption{Capture d'�cran Jasper Report Server}
\label{fig:capture_jasper}
\end{figure}
\begin{figure}[!ht]
\includegraphics[scale=1, width = 16cm, height = 12cm ]{fig/config_central.png}
\caption{Capture d'�cran Configuration centrale}
\label{fig:capture_config}
\end{figure}
\begin{figure}[!ht]
\includegraphics[scale=1, width = 20cm, height = 12cm ]{fig/timesheet.png}
\caption{Capture d'�cran timesheet}
\label{fig:capture_timesheet}
\end{figure}
\clearpage


%!TEX encoding = IsoLatin

%
% Chapitre "Bibliographie"
%

\begin{thebibliographyUL}{99} % remplacer le "{9}" par "{99}" lorsque le nombre de references
                              % requiert 2 caracteres (>= 10 references)

\bibitem{REF01} Wanted Technologies. \emph{A propos de nous}, [En ligne]. \url{https://www.wantedanalytics.com/fr/a-propos/a-propos-de-nous} (Page consult�e le 24 aout 2014)
 \end{thebibliographyUL}







%   Annexes
%\appendix
%\input{tex/liste_sig_acr}

\end{document}
% Fin du document

