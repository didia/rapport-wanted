%!TEX encoding = IsoLatin

%
% Chapitre "Reflexion th�orique"
%

\chapter{Reflexion sur la formation th�orique}
\label{s:reflexion_theorique}

Ma reflexion sur la formation th�orique reste presque identique � celle que j'ai fait lors de mon stage pr�c�dent. En effet, la formation th�orique re�ue � l'universit� a �t� en majeure partie tr�s utile dans mon stage. S'il y a une recommandation que je peux faire au directeur de programme, ce sera d'essayer de trouver des voies et moyens pour promouvoir et encourager des activit�s pratiques au sein du programme d'informatique et du g�nie logiciel pendant lesquelles les �tudiants pourront appliquer toutes les th�ories re�ues en classe. 

J'ai remarqu� que l'enseignement th�orique � l'Universit� Laval est de tr�s bon niveau. Cependant la pratique de cette derni�re est l� o� le b�t blesse. Est-ce peut-�tre parce que l'universit� est trop orient�e vers la recherche?. Toutefois je me dois d'avouer qu'au fur et � mesure que j'avance dans mon programme,  je remarque que les cours ont tendance � �tre de plus en plus pratique � l'instar des cours comme Qualit� et M�trique du Logiciel, Design III ou encore Architecture Logicielle et D�veloppement Web. C'est donc peut-�tre un bon signe.
