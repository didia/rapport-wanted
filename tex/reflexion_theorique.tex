%!TEX encoding = IsoLatin

%
% Chapitre "Reflexion th�orique"
%

\chapter{Reflexion sur la formation th�orique}
\label{s:reflexion_theorique}

La formation th�orique re�ue � l'universit� a �t� en majeure partie tr�s utile dans mon stage. S'il y a une recommandation que je peux faire au directeur de programme, ce sera d'essayer de trouver des voies et moyens pour promouvoir et encourager des activit�s pratiques au sein du programme d'informatique et du g�nie logiciel pendant lesquelles les �tudiants pourront appliquer toutes les th�ories re�ues en classe. 

J'ai remarque que l'enseignement th�orique � l'Universit� Laval est de tr�s bon niveau. Cependant la pratique de cette derni�re est l� o� le b�t blesse. Or une th�orie sans pratique est certaine de se retrouver dans les oublis. D'o� j'encouragerai le d�partement � promouvoir les activit�s pratiques de programmation telles que les hackatons ou encore que nos cours proposent des projets de session de haut niveau de programmation � l'instar du cours de G�nie Logiciel Orient� Object (GLO-2004) qui le fait si bien d�j�. 

D'ici l� , j'ose esp�rer que le cours de d�veloppement mobile que je vais suivre probablement � la session d'hiver sera � la hauteur de mes attentes, car apr�s avoir appris beaucoup sur le d�veloppement web au sein de Wanted, j'aimerais bien orienter ma carri�re vers la nouvelle vague de l'internet des objects \cite{REF02}. Et le d�veloppement mobile est le point de d�part qu'il me faut. 



