%!TEX encoding = IsoLatin

%
% Chapitre "Introduction"
%

\chapter{Conclusion}
\label{s:conclusion}
Apr�s mon interview, j?�tais convaincu que mon stage serait une exp�rience enrichissante. Mon superviseur de stage, Mr Gaetan avait r�ussi en m?en persuader. A la fin du stage, je ne peux qu?affirmer que mon intuition �tait bien correcte!
En effet, avant mon stage, j?avais suivi le cours d?Introduction au processus g�nie logiciel. De ce cours, j?avais appris diff�rentes �tapes principales du developpement logiciel comme la phase de r�quis, d?analyse, d?implementation, d?integration, de tests, etc..
Cependant pendant mon stage, au d�but, j?avais clairement sous estim� l?importance de la phase d?analyse dans le d�veloppement logiciel.
Toutefois, grace � l?�quipe enti�re, j?ai su d�couvrir l?importance de cette phase analytique au fil du temps et je m?y suis donn� � coeur joie.
Tous ces concepts je les avais bien maitris�s, pendant mon stage je les ai mis en pratique, je sais l?importance de l?analyse dans une conception logicielle et je sais comment le faire pratiquemet.
Mon developpement en java, en a aussi beaucoup profit�. Et le plus evidemment, c?est l?efficacit� que j?ai maintenant avec l?outil Eclipse. J?ai appris beaucoup des raccourcis et beaucoup de hacks qui ont augment� ma productivit� avec Eclipse.
 
