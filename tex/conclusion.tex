%!TEX encoding = IsoLatin

%
% Chapitre "Conclusion "
%

\chapter{Conclusion}
\label{s:conclusion}
Apr�s mon interview, j'�tais convaincu que mon stage serait une exp�rience enrichissante. Mon superviseur de stage, Mr Gaetan Corneau avait r�ussi en m'en persuader. A la fin du stage, je ne peux qu'affirmer que mon intuition �tait bien correcte!
En effet, avant mon stage, j'avais suivi le cours d'introduction aux processus de g�nie logiciel. De ce cours, j'avais appris diff�rentes �tapes principales du d�veloppement logiciel comme la phase de requis, d'analyse, d'implementation, d'integration, de tests, etc..
Durant mon stage, j'ai eu l'opportunit� de mettre pratique les acquis de ce cours et de me familiariser avec le processus de d�veloppement agile.
Mon aisance avec le langage de programmation Java s'est clairement am�lior�e gr�ce � l'aide pr�cieuse que toute mon �quipe m'a apport�e.
Maintenant, je vais continuer mon programme de baccalaur�at et continuer � apprendre et acqu�rir les connaissances n�cessaires qui vont me permettre de m'orienter vers le d�veloppement web et l'internet des objects qui sont les champs qui me fascinent pour l'instant.
 
