% !TEX encoding = UTF-8 Unicode

%
% Chapitre "Introduction"
%

\chapter{Introduction}
\label{s:intro}

En été 2014, je faisais déjà mon premier stage en Informatique à Wanted Technologies travaillant sur les outils internes pour la génération de rapport ainsi que quelques corrections des bugs dans l'API. À l'issu de ce stage riche en expérience inoubliable, j'étais convaincu que mon choix de cursus académique correspondait bien à ce que je voulais faire dans la vie.
Cet été je suis retourné dans la meme entreprise, à l'occurrence Wanted Technologies mais cette fois, dans une autre équipe, l'équipe User Interface (UI) . C'est donc de cette experience vécue dans l'équipe UI qu'il sera question dans les prochaines lignes de ce rapport. 


\section{Wanted Technologies}
\label{about_wanted}

WANTED Technologies est une entreprise informatique créé en 1999 qui fournit de l'information de veille commerciale en temps réel pour le marché du recrutement. La société a son siège social à Québec, au Canada, et maintient une filiale américaine ayant ses bureaux principaux à New York. Elle accumule, depuis octobre 2002, les détails associés aux offres d'emploi et maintient actuellement une base de données surpassant un milliard d'offres d'emploi uniques.
Ses clients qui proviennent des secteurs tels que  les ressources humaines, les services de recrutement, les médias et les gouvernements utilisent WANTED Analytics, son logiciel principal,  pour identifier et prioriser les pistes de vente, cerner les tendances économiques, analyser les activités de la concurrence, estimer les conditions économiques futures ainsi qu'identifier des candidats pour des postes difficiles à combler.  \cite{REF01}

