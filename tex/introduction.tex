% !TEX encoding = IsoLatin

%
% Chapitre "Introduction"
%

\chapter{Introduction}
\label{s:intro}

En �t� 2014, je faisais d�j� mon premier stage en Informatique � Wanted Technologies travaillant sur les outils internes pour la g�n�ration de rapport ainsi que quelques corrections des bugs dans l'API. � l'issu de ce stage riche en exp�rience inoubliable, j'�tais convaincu que mon choix de cursus acad�mique correspondait bien � ce que je voulais faire dans la vie.
Cet �t� je suis retourn� dans la meme entreprise, � l'occurrence Wanted Technologies mais cette fois, dans une autre �quipe, l'�quipe User Interface (UI) . C'est donc de cette experience v�cue dans l'�quipe UI qu'il sera question dans les prochaines lignes de ce rapport. 


\section{Wanted Technologies}
\label{about_wanted}

WANTED Technologies est une entreprise informatique cr�� en 1999 qui fournit de l'information de veille commerciale en temps r�el pour le march� du recrutement. La soci�t� a son si�ge social � Qu�bec, au Canada, et maintient une filiale am�ricaine ayant ses bureaux principaux � New York. Elle accumule, depuis octobre 2002, les d�tails associ�s aux offres d'emploi et maintient actuellement une base de donn�es surpassant un milliard d'offres d'emploi uniques.
Ses clients qui proviennent des secteurs tels que  les ressources humaines, les services de recrutement, les m�dias et les gouvernements utilisent WANTED Analytics, son logiciel principal,  pour identifier et prioriser les pistes de vente, cerner les tendances �conomiques, analyser les activit�s de la concurrence, estimer les conditions �conomiques futures ainsi qu'identifier des candidats pour des postes difficiles � combler.  \cite{REF01}

